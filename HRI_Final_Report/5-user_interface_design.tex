\section{User Interface Design}
\label{sec:user_interface_design}

To emulates having a controller that can control only a subset of the degrees of freedom of the robot in each mode, we had users navigate a two-dimensional space with a one dimensional controller. For each time step, the robot had a specific mode, denoted by the orientation of the yellow line at the center of the robot (blue square)~(\figref{fig:maps}). When the line is horizontal, the robot can only move in the horizontal direction. In this mode, the up and down arrow keys would map to left and right movement respectively. Likewise, when the line is vertical the robot can move vertically. In this mode, the up and down arrow keys would map to up and down movement respectively. The user could switch between the modes using the spacebar.

Our interface also implements all the three assistance types (see~\sref{sec:related_work}). With the automatic assistance type, we introduce the concept of \textit{optimal zones} which are sections of the map where one mode is more optimal than the other. These optimal zones are determined by the optimality map (see~\sref{sec:background}). An intuitive example is when navigating a vertical hallway, being in vertical mode would lead to fewer number of mode switches compared to being in horizontal mode. This is because the user would eventually switch to vertical to continue moving down the hallway, making there be one more mode switch compared to beginning with vertical mode.
