\def\year{2018}\relax
%File: formatting-instruction.tex
\documentclass[letterpaper]{article} %DO NOT CHANGE THIS
\usepackage{aaai18}  %Required
\usepackage{times}  %Required
\usepackage{helvet}  %Required
\usepackage{courier}  %Required
\usepackage{url}  %Required
\usepackage{graphicx}  %Required
\frenchspacing  %Required
\setlength{\pdfpagewidth}{8.5in}  %Required
\setlength{\pdfpageheight}{11in}  %Required

\usepackage{algorithm}
\usepackage{algorithmicx}
\usepackage[noend]{algpseudocode}
\usepackage{amsmath}
\usepackage{amssymb}  % assumes amsmath package installed
\usepackage{xpatch}
\usepackage{soul}
%\usepackage[linesnumbered,ruled]{algorithm2e}
%%\usepackage{bbm}
\usepackage{graphicx}
\usepackage{import}
\usepackage{url}
\usepackage{hyperref}
\usepackage[usenames,dvipsnames]{color}
\usepackage{pgfplots}
\usepackage[per-mode=symbol]{siunitx}
\usepackage[caption=true,labelformat=parens]{subfig}
\usepackage{tikz}
\usetikzlibrary{fit,calc}
\newcommand*{\tikzmk}[1]{\tikz[remember picture,overlay,] \node (#1) {};\ignorespaces}
%define a boxing command, argument = colour of box
\newcommand{\boxit}[1]{\tikz[remember picture,overlay]{\node[fill=#1,opacity=.25,fit={($(A)-(0.1,0.5\baselineskip)$)($(B)-(0.64\columnwidth,0.0)$)}] {};}\ignorespaces}

\newcommand{\boxitone}[1]{\tikz[remember picture,overlay]{\node[fill=#1,opacity=.25,yshift=2pt,fit={($(A)-(0.44,0.5\baselineskip)$)($(B)+(0.892\columnwidth,0.0)$)}] {};}\ignorespaces}

\newcommand{\boxittwo}[1]{\tikz[remember picture,overlay]{\node[fill=#1,opacity=.25,yshift=7pt,xshift=-120pt,fit={($(A)-(0.0,1.2\baselineskip)$)($(B)+(1.33\columnwidth,2.6)$)}] {};}\ignorespaces}

\newcommand{\boxitthree}[1]{\tikz[remember picture,overlay]{\node[fill=#1,opacity=.25,yshift=99pt,fit={($(A)-(0.415,0.5\baselineskip)$)($(B)+(0.89\columnwidth,0.8)$)}] {};}\ignorespaces}

\newcommand{\boxitfour}[1]{\tikz[remember picture,overlay]{\node[fill=#1,opacity=.25,xshift=100pt,fit={($(A)-(0.95,0.5\baselineskip)$)($(B)-(0.68\columnwidth,0.0)$)}] {};}\ignorespaces}

%define some colours according to algorithm parts (or any other method you like)
\colorlet{pink}{red!40}
\colorlet{cyan}{cyan!60}

\usepackage{color}
\usepackage{xspace}
\usepackage{enumitem}
\usepackage{xcolor}
\usepackage{sidecap}

% set colors for code background and highlighting
\definecolor{codehighlight}{rgb}{0.95,0.8,0.8}
\definecolor{codebackground}{rgb}{0.95,0.95,0.95}

\definecolor{complexbox}{rgb}{0.42,0.83,0.84}
\definecolor{trivialbox}{rgb}{0.98,0.64,0.61}
\definecolor{complexscatter}{rgb}{0.54,0.059,0}
\definecolor{trivialscatter}{rgb}{0.0039,0.09,0.55}
%%\usepackage{texcomp, libertine}

\captionsetup{belowskip=0pt}

\definecolor{orange}{rgb}{0.8,0.4,0}
\definecolor{mylink}{RGB}{18,68,115}
\definecolor{darkgreen}{rgb}{0.3,0.6,0.3}

\DeclareCaptionFont{green}{darkgreen}
\algrenewcommand{\alglinenumber}[1]{\footnotesize\textrm{#1:}}

% Insert hyperlinks in the document
\hypersetup{letterpaper,bookmarksopen,bookmarksnumbered,
pdfpagemode=UseOutlines,
colorlinks=true,
linkcolor=blue,
anchorcolor=blue,
citecolor=blue,
filecolor=blue,
menucolor=blue,
urlcolor=blue,
pdfinfo={
Title={Effective Footstep Planning for Humanoids Using Homotopy-Class Guidance},
Author={Vinitha Ranganeni, Oren Salzman, Maxim Likhachev}},
pdfproducer={LaTeX},
pdfcreator={pdfLaTeX},
pdfsubject={Robotics; Motion Planning},
pdfkeywords={Footstep Planning; Heuristics; Humanoids; Homotopy Classes}
}

% Algorithm commands
% \algnewcommand\algorithmicforeach{
%   \textbf{for each}}
%   \algdef{S}[FOR]{ForEach}[1]{\algorithmicforeach\ #1\ 
% \algorithmicdo}

\newenvironment{colorline}[1][]{
	\algrenewcommand{\algline}[1]{\textcolor{blue}}
}

% Hypothesis
\newcounter{hypothesis}[section]
\newenvironment{hypothesis}[1][]
{ \refstepcounter{hypothesis}\par\medskip
\textbf{H.\thehypothesis} #1 \rmfamily}{\medskip}

% Questions
\newcounter{question}[section]
\newenvironment{question}[1][]
{ \refstepcounter{question}\par\smallskip
\textbf{Q.\thequestion} #1 \rmfamily}{\vspace{0.5ex}}

% Sets
\newcounter{set}[section]
\newenvironment{set}[1][]
{ \refstepcounter{set}\par\vspace{0.3ex}
\textbf{S.\theset} #1 \rmfamily}{\vspace{0.3ex}}

% Color patches in captions
\definecolor{planned}{HTML}{31a354}
\definecolor{random}{HTML}{636363}
\definecolor{learned}{HTML}{a6611a}
\definecolor{basic}{HTML}{756bb2}
\definecolor{pcrrt}{HTML}{3182bd}
\definecolor{brrt}{HTML}{e6550d}
\newcommand{\CaptionLine}[1]{\raisebox{2pt}{\tikz{\draw[-,#1, solid, line width=0.9pt](0,0) -- (2mm,0);}}}
\newcommand{\CaptionCircle}[1]{\tikz{\draw[#1,fill=#1] (0,0) circle (.5ex);}}

\makeatletter
\def\endthebibliography{%
  \def\@noitemerr{\@latex@warning{Empty `thebibliography' environment}}%
  \endlist
}
\makeatother

\import{./}{macros.tex}

%PDF Info Is Required:
% \pdfinfo{
% /Title (Effective Footstep Planning for Humanoids Using Homotopy-Class Guidance)
% /Author (Vinitha Ranganeni, Oren Salzman, Maxim Likhachev)}
\setcounter{secnumdepth}{2} 
\begin{document}
% The file aaai.sty is the style file for AAAI Press 
% proceedings, working notes, and technical reports.
%
\title{Effective Footstep Planning for Humanoids Using Homotopy-Class Guidance}
\author{Vinitha~Ranganeni, Oren~Salzman, Maxim~Likhachev\\
The Robotics Institute, Carnegie Mellon University\\
\{vrangane, osalzman\}@andrew.cmu.edu,~maxim@cs.cmu.edu\\
}
\maketitle

%%%%%%%%%%%%%%%%%%%%%%%%%%%%%%%%%%%%%%%%%%%%%%%%%%%%%%%%%%%%%%%%%%%%%%%%%%%%%%%%
\begin{abstract}
Planning the motion for humanoid robots is a computationally-complex task due to the high dimensionality of the system. Thus, a common approach is to first plan in the low-dimensional space induced by the robot's feet---a task referred to as \emph{footstep planning}. This low-dimensional plan is then used to guide the full motion of the robot. One approach that has proven successful in footstep planning is using search-based planners such as \astar and its many variants. To do so, these search-based planners have to be endowed with effective heuristics to efficiently guide them through the search space. However, designing effective heuristics is a time-consuming task that requires the user to have good domain knowledge. Thus, our goal is to be able to effectively plan the footstep motions taken by a humanoid robot while obviating the burden on the user to carefully design local-minima free heuristics. To this end, we propose to use user-defined homotopy classes in the workspace that are intuitive to define. These homotopy classes are used to automatically generate heuristic functions that efficiently guide the footstep planner. 
We compare our approach for footstep planning with a standard approach that uses a heuristic common to footstep planning.
In simple scenarios, the performance of both algorithms is comparable.
However, in more complex scenarios our approach allows for a speedup in planning of several orders of magnitude when compared to the standard approach.
\end{abstract}


%%%%%%%%%%%%%%%%%%%%%%%%%%%%%%%%%%%%%%%%%%%%%%%%%%%%%%%%%%%%%%%%%%%%%%%%%%%%%%%%
%% Introduction
\import{./}{1-introduction.tex}

%% Related work
\import{./}{2-related_work.tex}

%% Background
\import{./}{3-algorithmic_background.tex}

%% Approach
\import{./}{4-approach.tex}

%% Results
\import{./}{5-experiments.tex}

%% Conclusion and Future work
\import{./}{6-conclusion.tex}

%% Acknowledgments
\import{./}{7-acknowledgments.tex}

%%\addtolength{\textheight}{-12cm}   % This command serves to balance the column lengths
                                  % on the last page of the document manually. It shortens
                                  % the textheight of the last page by a suitable amount.
                                  % This command does not take effect until the next page
                                  % so it should come on the page before the last. Make
                                  % sure that you do not shorten the textheight too much.

%%%%%%%%%%%%%%%%%%%%%%%%%%%%%%%%%%%%%%%%%%%%%%%%%%%%%%%%%%%%%%%%%%%%%%%%%%%%%%%%



%%%%%%%%%%%%%%%%%%%%%%%%%%%%%%%%%%%%%%%%%%%%%%%%%%%%%%%%%%%%%%%%%%%%%%%%%%%%%%%%



%%%%%%%%%%%%%%%%%%%%%%%%%%%%%%%%%%%%%%%%%%%%%%%%%%%%%%%%%%%%%%%%%%%%%%%%%%%%%%%%


%%%%%%%%%%%%%%%%%%%%%%%%%%%%%%%%%%%%%%%%%%%%%%%%%%%%%%%%%%%%%%%%%%%%%%%%%%%%%%%%

\bibliography{references}
\bibliographystyle{aaai}

\end{document}
