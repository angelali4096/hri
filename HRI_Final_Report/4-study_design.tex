\section{Study Design}
\label{sec:approach}

In our user study, we asked each participant to teleoperate a robot (blue box) and navigate it to a goal location (green box)in a 2-dimensional map using the mode-switching interface we provided.\\

Our hypotheses are as follows:
\begin{enumerate}
    \item [] \begin{hypothesis} Users will only trust robot’s automatic assistance types on fully-visible environments.\label{hyp:h1} \end{hypothesis}
    \item [] \begin{hypothesis} Users will trust robot’s automatic and forced assistance types on partially-visible environments.\label{hyp:h2} \end{hypothesis}
    \item [] \begin{hypothesis} There will be an overall decrease in time taken to reach the goal when the robot provides forced or automatic assistance versus manual.\label{hyp:h3} \end{hypothesis}
\end{enumerate}

We designed our user study to manipulate two independent variables: visibility and assistance types (see~\sref{sec:related_work}). Visibility had two levels: partial and full. The partially visibile only allowed the user to see a portion of the map as they moved the robot~(\figref{fig:partial_map}). In contrast, the fully visible map allowed the user to see the entire map as the moved the robot~(\figref{fig:full_map}). 

We developed 3$\times$2 factorial design to test our independent variables. There were 6 total conditions, with 21 users in each condition. We used a within-subjects design and gave users a practice map~(\figref{fig:diag_map}) before running our study to avoid the expertise effect.
